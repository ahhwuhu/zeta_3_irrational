% In this file you should put the actual content of the blueprint.
% It will be used both by the web and the print version.
% It should *not* include the \begin{document}
%
% If you want to split the blueprint content into several files then
% the current file can be a simple sequence of \input. Otherwise It
% can start with a \section or \chapter for instance.

\chapter{Introduction}

证明一个实数$\xi$是无理数, 我们通常采用构造非零序列$\{a_n + b_n\xi\}$趋于$0$, 当$n \rightarrow \infty$时, 其中$a_n, b_n \in \mathbb{Z}$.

因为如果$\xi$是有理数$\frac{p}{q}$, 则上述序列有下界$\frac{1}{q}$.

\chapter{Proof}

\begin{definition}\label{Legendre_poly}
    \[ P_n(x):=\frac{1}{n!}\frac{d^n}{dx^n}[x^n(1-x)^n] \]
\end{definition}

\begin{lemma}\label{Legendre_poly_is_int}
    \[ P_n(x)=\sum\limits_{k=0}^{n}(-1)^k\binom{n}{k}\binom{n+k}{n}x^k \]
\end{lemma}

\begin{lemma}\label{Legendre_poly_sym}
    \[ P_n(1-x)=(-1)^nP_n(x) \]
\end{lemma}

\begin{lemma}\label{Legendre_poly_integral}
    For all $n \in \mathbb{N}$ and $f : [0,1] \rightarrow \mathbb{R}$ of class $C^n$, one has
    \[ \int_{0}^{1}P_n(x)f(x) \, dx =\frac{(-1)^n}{n!} \int_{0}^{1} x^n(1-x)^n\frac{d^nf}{dx^n} \, dx \]
\end{lemma}

\begin{lemma}\label{integral_zeta_3}
    \[ -\int_{0}^{1}\int_{0}^{1} \frac{{\rm ln}(xy)}{1-xy} \, dx \, dy = 2\xi(3) \]
\end{lemma}

\begin{lemma}\label{I_rr}
    for all integers $r > 0$
    \[ -\int_{0}^{1}\int_{0}^{1} x^ry^r\frac{1}{1-xy} \, dx \, dy = \sum\limits_{m = 1}^{\infty}\frac{1}{(m+r)^2} \]
\end{lemma}

\begin{lemma}\label{interchange_of_limits_sums_derivatives}
    111
\end{lemma}

\begin{lemma}\label{J_rr}
    \use{I_rr, interchange_of_limits_sums_derivatives}
    for all integers $r > 0$
    \[ -\int_{0}^{1}\int_{0}^{1} x^ry^r\frac{{\rm ln}(xy)}{1-xy} \, dx \, dy = 2\xi(3) - 2 \sum\limits_{m = 1}^{r}\frac{1}{m^3} \]
\end{lemma}

\begin{lemma}\label{I_rs}
    Let $r$ and $s$ be non-negative integers, with $r \neq s$, then
    \[ -\int_{0}^{1}\int_{0}^{1} x^ry^s\frac{1}{1-xy} \, dx \, dy = \sum\limits_{m = 1}^{\infty}\frac{1}{m+r}\frac{1}{m+s} \]
\end{lemma}

\begin{lemma}\label{J_rs}
    \use{I_rs, interchange_of_limits_sums_derivatives}
    Let $r$ and $s$ be non-negative integers, with $r \neq s$, then
    \[ -\int_{0}^{1}\int_{0}^{1} x^ry^s\frac{{\rm ln}(xy)}{1-xy} \, dx \, dy = \frac{\sum\limits_{m = 1}^{r}\frac{1}{m^2} - \sum\limits_{m = 1}^{s}\frac{1}{m^2}}{r - s} \]
\end{lemma}

\begin{lemma}\label{d_r_3}
    For all $r \in \mathbb{N}^*$,
    \[ d_{r^3} = (d_r)^3 \]
\end{lemma}

\begin{lemma}\label{Jrr_linear_form}
    \use{J_rr, d_r_3}
    For all $r \in \mathbb{N}^*$,
    \[ J_rr = 2 \zeta(3) - \frac{z_r}{(d_r)^3} \]
    for some $z_r \in \mathbb{N}^*$.
\end{lemma}

\begin{lemma}\label{d_r_2}
    For all $r \in \mathbb{N}^*$,
    \[ d_{r^2} = (d_r)^2 \]
\end{lemma}

\begin{lemma}\label{Jrs_postive_rational}
    \use{J_rs, d_r_2, d_r_3}
    For all $r \in \mathbb{N}, r \neq s$,
    \[ J_rs = \frac{z_{rs}}{(d_r)^3}\]
    for some $z_{rs} \in \mathbb{N}^*$.
\end{lemma}

\begin{lemma}\label{one_var_substitution}
    \[ \int_{0}^{1} \frac{1}{1 - (1 - v)z} \, dz= -\frac{{\rm ln}v}{1 - v} \]
\end{lemma}

\begin{lemma}\label{two_var_substitution}
    \use{one_var_substitution}
    Given $s, t \in \mathbb{R}_{(0,1)}$, the following equality holds:
    \[ \int_{0}^{1} \frac{1}{1 - [1 - (1 - s)t]u} \, du = \int_{0}^{1} \frac{1}{[1 - (1 - u)s][1 - (1 - t)u]} \, du \]
\end{lemma}

\begin{definition}\label{J_n}
    \[ J_n := \int_{0}^{1} P_n(x)P_n(y)\frac{{\rm ln}(xy)}{1-xy} \, dx \, dy \]
\end{definition}

\begin{lemma}\label{J_n_integers_an_bn}
    \use{J_n, integral_zeta_3, Jrr_linear_form, Jrs_postive_rational}
    For some integers $a_n$ and $b_n$,
    \[ J_n = \frac{a_n}{d_n^3} + b_n\zeta(3) \]
\end{lemma}

\begin{lemma}\label{frac_partial_n}
    For $a \in \mathbb{R}_{(0,1)}$,
    \[ \frac{\partial^n}{\partial x^n}(\frac{1}{1-ax}) = \frac{n!a^n}{(1-ax)^n} \]
\end{lemma}

\begin{lemma}\label{bound}
    Let $D = \{(x,y,z)|x,y,z\in (0,1)\}$, then
    \[ \frac{x(1-x)y(1-y)z(1-z)}{[1-(1-z)x](1-yz)} < \frac{1}{30} \]
\end{lemma}

\begin{theorem}\label{zeta_3_irrational}
    $\zeta(3)$ is irrational.
\end{theorem}