% In this file you should put the actual content of the blueprint.
% It will be used both by the web and the print version.
% It should *not* include the \begin{document}
%
% If you want to split the blueprint content into several files then
% the current file can be a simple sequence of \input. Otherwise It
% can start with a \section or \chapter for instance.

\chapter{Introduction}

证明一个实数$\xi$是无理数, 我们通常采用构造非零序列$\{a_n + b_n\xi\}$趋于$0$, 当$n \rightarrow \infty$时, 其中$a_n, b_n \in \mathbb{Z}$.

因为如果$\xi$是有理数$\frac{p}{q}$, 则上述序列有下界$\frac{1}{q}$.

\chapter{Proof}

\begin{definition}\label{Legendre_poly}
    \[ P_n(x):=\frac{1}{n!}\frac{d^n}{dx^n}[x^n(1-x)^n] \]
\end{definition}

\begin{lemma}\label{Legendre_poly_is_int}
    \[ P_n(x)=\sum\limits_{k=0}^{n}(-1)^k\binom{n}{k}\binom{n+k}{n}x^k \]
\end{lemma}
\begin{proof}
    \leanok
    \begin{align*}
        \frac{1}{n!}\frac{d^n}{dx^n}[x^n(1-x)^n] &= \frac{1}{n!}\frac{d^n}{dx^n}[(x-x^2)^n]\\
        &= \frac{1}{n!}\frac{d^n}{dx^n}[\sum\limits_{k=0}^{n} \binom{n}{k}(-1)^k x^{n-k}x^{2k}]\\
        &= \frac{1}{n!}\frac{d^n}{dx^n}[\sum\limits_{k=0}^{n} \binom{n}{k}(-1)^k x^{n+k}]\\
        &= \frac{1}{n!}\sum\limits_{k=0}^{n} \binom{n}{k}(-1)^k \frac{d^n}{dx^n}[x^{n+k}]\\
        &= \frac{1}{n!}\sum\limits_{k=0}^{n} \binom{n}{k}(-1)^k \frac{(n+k)!}{k!}x^{k}\\
        &= \sum\limits_{k=0}^{n} \binom{n}{k}(-1)^k \frac{1}{n!}\frac{(n+k)!}{k!}x^{k}\\
        &= \sum\limits_{k=0}^{n} (-1)^k \binom{n}{k}\binom{n+k}{n}x^{k}
    \end{align*}
\end{proof}

\begin{lemma}\label{Legendre_poly_sym}
    \[ P_n(1-x)=(-1)^nP_n(x) \]
\end{lemma}
\begin{proof}
    \leanok
    By definition.
\end{proof}

\begin{lemma}\label{Legendre_poly_mul_frac_integral}
    For all $ 0 < a < 1$, one has
    \[ \int_{0}^{1}\frac{P_n(x)}{1 - ax} \, dx =(-1)^n \int_{0}^{1} x^n(1-x)^n\frac{a^n}{(1 - ax)^{n+1}} \, dx \]
\end{lemma}
\begin{proof}
    \leanok
    By induction.
\end{proof}

\begin{lemma}\label{integral_zeta_3}
    \[ J_{00} := -\int_{0}^{1}\int_{0}^{1} \frac{{\rm ln}(xy)}{1-xy} \, dx \, dy = 2\zeta(3) \]
\end{lemma}

\begin{lemma}\label{I_rr}
    for all integers $r > 0$
    \[ I_{rr} := -\int_{0}^{1}\int_{0}^{1} x^ry^r\frac{1}{1-xy} \, dx \, dy = \sum\limits_{m = 1}^{\infty}\frac{1}{(m+r)^2} \]
\end{lemma}

\begin{lemma}\label{J_rr}
    \uses{I_rr}
    for all integers $r > 0$
    \[ J_{rr} := -\int_{0}^{1}\int_{0}^{1} x^ry^r\frac{{\rm ln}(xy)}{1-xy} \, dx \, dy = 2\zeta(3) - 2 \sum\limits_{m = 1}^{r}\frac{1}{m^3} \]
\end{lemma}

\begin{lemma}\label{I_rs}
    Let $r$ and $s$ be non-negative integers, with $r \neq s$, then
    \[ I_{rs} := -\int_{0}^{1}\int_{0}^{1} x^ry^s\frac{1}{1-xy} \, dx \, dy = \sum\limits_{m = 1}^{\infty}\frac{1}{m+r}\frac{1}{m+s} \]
\end{lemma}

\begin{lemma}\label{J_rs}
    \uses{I_rs}
    Let $r$ and $s$ be non-negative integers, with $r \neq s$, then
    \[ J_{rs} := -\int_{0}^{1}\int_{0}^{1} x^ry^s\frac{{\rm ln}(xy)}{1-xy} \, dx \, dy = \frac{\sum\limits_{m = 1}^{r}\frac{1}{m^2} - \sum\limits_{m = 1}^{s}\frac{1}{m^2}}{r - s} \]
\end{lemma}

\begin{lemma}\label{d_r_3}
    \leanok
    For all $r \in \mathbb{N}^*, d_n$ is lcm of $\{1, 2, \ldots, n\}.$ 
    \[ d_{r^3} = (d_r)^3 \]
\end{lemma}
\begin{proof}
    \leanok
    By prime factor expand.
\end{proof}

\begin{lemma}\label{Jrr_linear_form}
    \uses{J_rr, d_r_3}
    For all $r \in \mathbb{N}^*$,
    \[ J_{rr} = 2 \zeta(3) - \frac{z_r}{(d_r)^3} \]
    for some $z_r \in \mathbb{Z}$.
\end{lemma}
\begin{proof}
    \leanok
    By computing.
\end{proof}

\begin{lemma}\label{d_r_2}
    \leanok
    For all $r \in \mathbb{N}^*, d_n$ is lcm of $\{1, 2, \ldots, n\}.$ 
    \[ d_{r^2} = (d_r)^2 \]
\end{lemma}
\begin{proof}
    By prime factor expand.
\end{proof}

\begin{lemma}\label{Jrs_postive_rational}
    \uses{J_rs, d_r_2, d_r_3}
    For all $r \in \mathbb{N}, r \neq s$,
    \[ J_{rs} = \frac{z_{rs}}{(d_r)^3}\]
    for some $z_{rs} \in \mathbb{Z}$.
\end{lemma}
\begin{proof}
    \leanok
    By computing.
\end{proof}

\begin{lemma}\label{one_var_substitution}
    \[ \int_{0}^{1} \frac{1}{1 - (1 - x)z} \, dz= -\frac{{\rm ln}x}{1 - x} \]
\end{lemma}
\begin{proof}
    \leanok
    Substitute $y = (1 - x)z$ in the integral, and we also have $dy = (1-x)dz$. Then we deduce that
    \begin{align*}
        \int_{0}^{1} \frac{1}{1 - (1 - x)z} \, dz =& \int_{0}^{1-x} \frac{1}{(1 - y)(1-x)} \, dy \\
        =&\frac{1}{1-x}\int_{0}^{1-x} \frac{1}{1 - y} \, dy\\
        =&\frac{1}{1-x}[-{\rm ln}(1-y)]_0^{1-x} \\
        =&\frac{1}{1-x}[-{\rm ln}(x) + {\rm ln}(1)] \\
        =&-\frac{{\rm ln}x}{1 - x}
    \end{align*}
\end{proof}

\begin{lemma}\label{two_var_substitution}
    \uses{one_var_substitution}
    Given $s, t \in \mathbb{R}_{(0,1)}$, the following equality holds:
    \[ \int_{0}^{1} \frac{1}{1 - [1 - (1 - s)t]u} \, du = \int_{0}^{1} \frac{1}{[1 - (1 - u)s][1 - (1 - t)u]} \, du \]
\end{lemma}
\begin{proof}
    \leanok
    Let $x = (1 - s)t$ in the previous lemma, then we can get 
    \[ \int_{0}^{1} \frac{1}{1 - [1 - (1 - s)t]u} \, du = -\frac{{\rm ln}[(1 - s)t]}{1 - (1 - s)t}\]
    We can also find that
    \[ \frac{1}{[1 - (1 - u)s][1 - (1 - t)u]} = \frac{1}{1 - (1 - s)t}\left[\frac{s}{1 - (1 - u)s} + \frac{1-t}{1 - (1 - t)u} \right] \]
    Then we can see the integral on the right side of the equal:
    \begin{align*}
        &\int_{0}^{1} \frac{1}{[1 - (1 - u)s][1 - (1 - t)u]} \, du \\
        =&\frac{1}{1 - (1 - s)t}\int_{0}^{1} \left[\frac{s}{1 - (1 - u)s} + \frac{1-t}{1 - (1 - t)u} \right] \, du \\
        =&\frac{1}{1 - (1 - s)t}\int_{0}^{1} \left[\frac{s}{1 - us} + \frac{1-t}{1 - (1 - t)u} \right] \, du \\
        =&\frac{1}{1 - (1 - s)t}\left[-{\rm ln}(1-su)-{\rm ln}(1-(1-t)u) \right]_{0}^{1}\\
        =&\frac{1}{1 - (1 - s)t}\left[-{\rm ln}(1-s)-{\rm ln}(t)+{\rm ln}(1)-{\rm ln}(1) \right]\\
        =&\frac{1}{1 - (1 - s)t}\left[-{\rm ln}(1-s)-{\rm ln}(t) \right]\\
        =&-\frac{{\rm ln}[(1 - s)t]}{1 - (1 - s)t}
    \end{align*}
    Immediately we can see the lemma.
\end{proof}

\begin{definition}\label{J_n}
    \[ J_n := - \int_{0}^{1}\int_{0}^{1} P_n(x)P_n(y)\frac{{\rm ln}(xy)}{1-xy} \, dx \, dy \]
\end{definition}

\begin{lemma}\label{J_n_integers_an_bn}
    \uses{J_n, integral_zeta_3, Jrr_linear_form, Jrs_postive_rational, Legendre_poly_is_int}
    For some integers $a_n$ and $b_n$,
    \[ J_n = \frac{a_n}{d_n^3} + b_n\zeta(3) \]
\end{lemma}
\begin{proof}
    \leanok
    Since $P_n(x) \in \mathbb{Z}[x]$. Suppose $P_n(x) = \sum\limits_{k=0}^{n}a_kx^k$, where $a_k \in \mathbb{Z}$.\\
    Then 
    \begin{align*}
        J_n &= -\int_{0}^{1}\int_{0}^{1} P_n(x)P_n(y)\frac{{\rm ln}(xy)}{1-xy} \, dx \, dy \\
        &= -\int_{0}^{1}\int_{0}^{1} \sum\limits_{i=0}^{n}a_ix^i \sum\limits_{j=0}^{n}a_jy^j \frac{{\rm ln}(xy)}{1-xy} \, dx \, dy\\
        &= \sum\limits_{i=0}^{n}\sum\limits_{j=0}^{n}a_i a_j -\int_{0}^{1}\int_{0}^{1} x^i y^j \frac{{\rm ln}(xy)}{1-xy} \, dx \, dy\\
        &= \sum\limits_{i=0}^{n}\sum\limits_{j=0}^{n}a_i a_j J_{ij}\\
    \end{align*}
    We have $J_{rr}$ and $J_{rs} \in \mathbb{Z}\zeta(3) + \frac{\mathbb{Z}}{d_n^3}$.\\
    So $J_n \in \mathbb{Z}\zeta(3) + \frac{\mathbb{Z}}{d_n^3}$.
\end{proof}

\begin{lemma}\label{frac_partial_n}
    For $a \in \mathbb{R}_{>0}$,
    \[ \frac{\partial^n}{\partial x^n}(\frac{1}{1-ax}) = \frac{n!a^n}{(1-ax)^n} \]
\end{lemma}
\begin{proof}
    \leanok
    By definition.
\end{proof}

\begin{lemma}\label{bound}
    Let $D = \{(x,y,z)|x,y,z\in (0,1)\}$, then
    \[ \frac{x(1-x)y(1-y)z(1-z)}{[1-(1-z)x](1-yz)} < \frac{1}{30} \]
\end{lemma}
\begin{proof}
    \leanok
    We have two inequalities
    \[ 1-(1-z)x = 1-x+xz \geqslant 2\sqrt{1-x}\sqrt{xz} \]
    and
    \[ 1-yz= 1-y+y-yz = 1-y+(1-z)y \geqslant 2\sqrt{1-y}\sqrt{(1-z)y} \]
    Then we can deduce that for $(x,y,z) \in D$,
    \begin{align*}
        \frac{x(1-x)y(1-y)z(1-z)}{[1-(1-z)x](1-yz)} \leqslant& \frac{x(1-x)y(1-y)z(1-z)}{2\sqrt{1-x}\sqrt{xz}\cdot2\sqrt{1-y}\sqrt{(1-z)y}}\\
        =&\frac{x(1-x)y(1-y)z(1-z)}{4\sqrt{1-x}\sqrt{x}\sqrt{z}\sqrt{1-y}\sqrt{1-z}\sqrt{y}}\\
        =&\frac{\sqrt{x}\sqrt{1-x}\sqrt{y}\sqrt{1-y}\sqrt{z}\sqrt{1-z}}{4}
    \end{align*}
    For $x\in (0,1)$, the max value of $\sqrt{x}\sqrt{1-x} = \sqrt{x(1-x)}$ is got at $x=\frac{1}{2}$. So do $y, z$.
    Then we have
    \begin{align*}
        \frac{x(1-x)y(1-y)z(1-z)}{[1-(1-z)x](1-yz)} \leqslant& \frac{1}{4}\cdot\sqrt{\frac{1}{2}}\sqrt{1-\frac{1}{2}}\cdot\sqrt{\frac{1}{2}}\sqrt{1-\frac{1}{2}}\cdot\sqrt{\frac{1}{2}}\sqrt{1-\frac{1}{2}} \\
        =& \frac{1}{32} < \frac{1}{30} 
    \end{align*}
\end{proof}

\begin{lemma}\label{Jn_abs_upper}
    \uses{bound, frac_partial_n, one_var_substitution, two_var_substitution, Legendre_poly_sym, Legendre_poly_mul_frac_integral}
    \[ J_n \leqslant (\frac{1}{30})^n\cdot 2\zeta(3) \]
\end{lemma}
\begin{proof}
    \begin{align*}
        J_n &= \int_{0}^{1}\int_{0}^{1} P_n(x)P_n(y)\frac{-{\rm ln}(xy)}{1-xy} \, dx \, dy \\
        &= \int_{0}^{1}\int_{0}^{1} P_n(x)P_n(y) \left[ \int_{0}^{1} \frac{1}{1 - (1 - xy)z} \, dz\right]\, dx \, dy \\
        &= \int_{0}^{1}\int_{0}^{1} P_n(1-x)P_n(y) \left[ \int_{0}^{1} \frac{1}{1 - (1 - (1-x)y)z} \, dz\right]\, dx \, dy \\
        &= \int_{0}^{1}\int_{0}^{1} (-1)^nP_n(x)P_n(y) \left[ \int_{0}^{1} \frac{1}{[1 - (1 - z)x][1 - (1 - y)z]} \, dz\right]\, dx \, dy \\
        &= \int_{0}^{1}\int_{0}^{1}\int_{0}^{1} \frac{P_n(x)P_n(1-y)}{[1 - (1 - z)x][1 - (1 - y)z]} \, dx \, dy \, dz \\
        &= \int_{0}^{1}\int_{0}^{1}\int_{0}^{1} \frac{P_n(x)P_n(y)}{[1 - (1 - z)x][1 - yz]} \, dx \, dy \, dz \\
        &= \int_{0}^{1} \left( \int_{0}^{1} \frac{P_n(x)}{1 - (1 - z)x} \, dx \int_{0}^{1} \frac{P_n(y)}{1 - yz} \, dy \right) \, dz \\
        &= \int_{0}^{1} \\
        &= \int_{0}^{1}\int_{0}^{1}\int_{0}^{1} \frac{\left(x-x^2\right)^n\left(y-y^2\right)^n\left(z-z^2\right)^n}{\{[1-(1-z) x](1-y z)\}^{n+1}} \, dx \, dy \, dz \\
        &= \int_{0}^{1}\int_{0}^{1}\int_{0}^{1} \frac{g(x, y, z)^n}{[1-(1-z)x](1-yz)} \, dx \, dy \, dz \\
        &< (\frac{1}{30})^n \int_{0}^{1}\int_{0}^{1}\int_{0}^{1} \frac{1}{[1-(1-z)x](1-yz)} \, dx \, dy \, dz \\
        &< (\frac{1}{30})^n \int_{0}^{1}\int_{0}^{1}\int_{0}^{1} \frac{1}{[1-(1-z)x](1-(1-y)z)} \, dx \, dy \, dz \\
        &= (\frac{1}{30})^n \int_{0}^{1}\int_{0}^{1}\int_{0}^{1} \frac{1}{1 - (1 - (1-x)y)z} \, dx \, dy \, dz \\
        &= (\frac{1}{30})^n \int_{0}^{1}\int_{0}^{1}\int_{0}^{1} \frac{1}{1 - (1 - xy)z} \, dx \, dy \, dz \\
        &= (\frac{1}{30})^n \int_{0}^{1}\int_{0}^{1} -\frac{{\rm ln}(xy)}{1 - xy} \, dx \, dy \\
        &= (\frac{1}{30})^n \cdot 2\zeta(3)
    \end{align*}
\end{proof}

\begin{lemma}\label{dn_asymptotic}
    Let $n$ be a positive integer. Define $\pi(n)$ as the number of primes less than (or equal to) $n$. Then, $d_n \leqslant n^{\pi(n)} \sim e^n$.
\end{lemma}
\begin{proof}
    use Prime Number Theorem. 
\end{proof}

\begin{theorem}\label{zeta_3_irrational}
    $\zeta(3)$ is irrational.
\end{theorem}
\begin{proof}
    \leanok
    \uses{J_n_integers_an_bn, Jn_abs_upper, dn_asymptotic}
    \[ 0 \neq |J_n| \leqslant (\frac{1}{30})^n\cdot 2\zeta(3) \]
    Then 
    \[ 0 < |\frac{a_n}{d_n^3} + b_n\zeta(3)| \leqslant (\frac{1}{30})^n\cdot 2\zeta(3) \]
    which means that 
    \[ 0 < |a_n + d_n^3 b_n\zeta(3)| \leqslant d_n^3(\frac{1}{30})^n\cdot 2\zeta(3) \] 
    又$d_n \leqslant n^{\pi(n)} \sim e^n$ and $e^3 < 21$, we have
    \[ 0 < |a_n + c_n\zeta(3)| < 21^n (\frac{1}{30})^n\cdot 2\zeta(3) = 2(\frac{7}{10})^n \zeta(3) \]
    where$c_n = d_n^3 b_n$ is integer.\\
    Assume $\zeta(3) = \frac{p}{q}, (p,q)=1$ and $p,q>0$. Then 
    \[ 0 < |qa_n + pc_n| < 2p (\frac{7}{10})^n \]
    So $n \rightarrow \infty, |qa_n + pc_n| \rightarrow 0$.\\
    Since $|qa_n + pc_n|$ is a integer, so $|qa_n + pc_n| \geqslant 1$. Contradiction! \\
    So $\zeta(3)$ is irrational!
\end{proof}